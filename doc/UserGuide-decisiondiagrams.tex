\documentclass[a4paper,11pt]{article}
\usepackage[ngerman,english]{babel}
\usepackage[latin1]{inputenc}
\usepackage{amssymb}
\usepackage{graphicx}
\usepackage{verbatim}
\usepackage{hyperref}
\usepackage{listings}
\usepackage{amsmath}
\usepackage{color}
\usepackage{qtree}
\usepackage{appendix}
\usepackage{float}
\usepackage{boxedminipage}
\usepackage{xspace}
\usepackage{tabularx}
\usepackage{amsthm}
\usepackage{ifthen}
\usepackage{tikz}
\usepackage{tikz-qtree}
\usepackage{verbatim}
\usepackage[ruled,vlined]{algorithm2e}
\usepackage{color}

\theoremstyle{definition}
\newtheorem{example}{Example}

\lstdefinelanguage{dlvhex}{
	keywords={not},
	otherkeywords={:-,[,],(,),{,},&,.,\,},
	keywordstyle=\textbf,
	sensitive=true,
	comment=[l]{\%},
	string=[b]",
	stringstyle=\color{gray}\textsf,
	showspaces=false,
	showstringspaces=false,
	showtabs=false
}

\newenvironment{mathprogram}[1][P]
	{
		% align at implication character
		\renewcommand{\ruleimplication}{&\leftarrow&}
		\newcommand{\ruledelimiter}{1}
		\begin{eqnarray*}
		#1 = \{
	}
	{
		\}
		\end{eqnarray*}
	}

\lstnewenvironment{program}[3][dlvhex]
	{
		\lstset{
			language=#1,
			basicstyle=\footnotesize,
			numberstyle=\footnotesize,
			caption={#2},
			captionpos=b,
			label={#3},
			%frame=single,
			emph={:=,(,),\{,\}},
			emphstyle=\textbf,
			mathescape=true
		}
	}
	{
	}

\lstnewenvironment{exampleprogram}[1][dlvhex]
	{
		\lstset{
			language=#1,
			basicstyle=\footnotesize,
			numberstyle=\footnotesize,
			captionpos=b,
			emph={:=,(,),\{,\}},
			emphstyle=\textbf,
			mathescape=true
		}
	}
	{
	}

\newcommand{\defneg}{not\xspace}
\newcommand{\ruleimplication}{\ensuremath{\leftarrow}}
\DeclareMathOperator{\OpDefNeg}{not}
\newcommand{\dlv}{\textsf{DLV}\xspace }
\newcommand{\hex}{\textsf{HEX}\xspace }
\newcommand{\dlvhex}{\textsf{dlvhex}\xspace }
\newcommand{\dotff}{\textsf{dot}\xspace }
\newcommand{\inlinecode}[1]{\textsf{#1}\xspace }
\newcommand{\true}{\emph{true}\xspace }
\newcommand{\false}{\emph{false}\xspace }
\newcommand*{\bigtimes}{\mathop{\raisebox{-.5ex}{\hbox{\huge{$\times$}}}}}

\lstdefinelanguage{dot}{
	keywords={digraph,graph},
	otherkeywords={->,;},
	keywordstyle=\textbf,
	sensitive=false,
	string=[b]",
	stringstyle=\color{gray}\textsf,
	showspaces=false,
	showstringspaces=false,
	showtabs=false
}

\newcommand{\DrawDD}
	{
		\tikzset{edge from parent/.style=
		  {draw,
		  ->,
		   edge from parent path={(\tikzparentnode) -- (\tikzchildnode)}}} 
		\tikzset{every node/.style={draw, rectangle, rounded corners=1mm, minimum width=1cm}}
		\tikzset{every leaf node/.style={draw, rectangle, rounded corners=0mm}}
		\tikzset{level distance=1.2cm}
		\tikzset{el/.style={auto,draw=none}}	% edge labels
	}
	

\begin{document}

	\title{decisiondiagramsplugin - User Guide}
	\date{April 03, 2010}
	\author{Christoph Redl\footnote{christoph.redl@student.tuwien.ac.at}}
	\maketitle

	The decisiondiagramsplugin adds support for working with decision diagrams to \dlvhex. In consists of a tool for conversion between the \dotff format and sets of facts as well as
	several operators for decision diagram modification and merging.
	
	The plugin was developed as part of the master's thesis \emph{Merging of Biomedical Decision Diagrams} \cite{CR10DD}.

	
	\section{File Format Conversions}
	\label{sec:Conversion}

		Decision diagrams can be stored in different file formats. While some of them are human-readable, others are better for automatic processing.
		
		\subsection{Supported Formats}

			\subsubsection{\dotff}

				The \dotff file format\footnote{\url{http://www.graphviz.org}} is intuitively readable and thus fit for being used as human readable format for representing decision diagrams.
				Additionally it is well suited for being visualized using the \emph{\dotff tools}.
			
				Consider the following decision diagram.
				
				\begin{minipage}[t]{1.00\textwidth}
					\label{fig:DecisionDiagram}
					\center
					\begin{tikzpicture}	
						\DrawDD
						\Tree
						[.\node{root};
							\edge node[el,auto=right]{$A < 10$};
							[.\node{$\mathit{case}_1$};
								\edge node[el,auto=right]{$B < 10$};
								\node{$\mathit{Class}_A$};
								\edge node[el]{else};
								\node{$\mathit{Class}_B$};
							]
							\edge node[el,auto=right]{$A > 20$};
							[.\node{$\mathit{case}_2$};
								\edge node[el,auto=right]{$B < 16$};
								\node{$\mathit{Class}_A$};
								\edge node[el]{else};
								\node{$\mathit{Class}_B$};
							]
							\edge node[el]{else};
							\node{$\mathit{Class}_C$};
						]
					\end{tikzpicture}
				\end{minipage}
				
				\noindent Then the following snippet shows a it's implementation as \dotff file.

				\begin{program}[dot]{the above decision diagram in \dotff format}{lst:DotDD}
	digraph G {
		root -> case1 ["A<10"];
		root -> case2 ["A>20"];
		root -> elsecase ["else"];
		root -> case3 ["else"];
		case1 -> case1a ["B<10"];
		case1 -> case1b ["else"];
		case2 -> case2a ["B<16"];
		case2 -> case2b ["else"];
		case1a ["ClassA"];
		case1b ["ClassB"];
		case2a ["ClassA"];
		case2b ["ClassB"];
		case3 ["ClassC"];
	}
				\end{program}

				Valid decision diagrams must
				\begin{itemize}
					\item have exactly one root node (which does not need to be explicitly mentioned, but which is implicitly identified by the fact that it has no ingoing edges)
					\item use only directed edges
					\item use only edges that are labeled either with
							$$\mathit{else}$$
						or with conditions of form
							$$X \circ Y$$
						where $X$ and $Y$ can be arbitrary strings and $\circ \in \{ <, <=, =, >, >= \}$ is an operator
					\item have leaf nodes that are labeled with arbitrary strings that encode the classification in this node
				\end{itemize}


			\subsubsection{Sets of Facts}

				However, \dlvhex cannot directly load this format because it's input must be a logic program. Thus a diagram must be represented using predicates.
				
				We define the following predicates:
				\begin{itemize}
					\item $\textit{root}(X)$ \\
						To define that some constant $X$ is defined as the root node
					\item $\textit{innernode}(X)$ \\
						To define some constant $X$ to be an inner node
					\item $\textit{leafnode}(X, Y)$ \\
						To define that some constant $X$ is a leaf node with label $Y$
					\item $\textit{conditionaledge}(X, Y, A, C, B)$ \\
						To define that a conditional edge with condition $A \circ B$ (where the operation $\circ$ is given by $C$) leads from node $X$ to $Y$
					\item $\textit{elseedge}(X, Y)$ \\
						To define that an unconditional edge goes from $X$ to $Y$
				\end{itemize}
				
				The above diagram can therefore be implemented as follows.
				
				\begin{program}[dlvhex]{the above decision diagram as \hex program}{lst:HexDD}
root(root).
innernode(case1).
innernode(case2).
leafnode(case3, "ClassC").
leafnode(case1a, "ClassA").
leafnode(case1b, "ClassB").
leafnode(case2a, "ClassA").
leafnode(case2b, "ClassB").
conditionaledge(root, case1, "A", "<", "10").
conditionaledge(root, case2, "A", ">", "20").
elseedge(root, case3).
conditionaledge(case1, case1a, "B", "<", "10").
elseedge(case1, case1b).
conditionaledge(case2, case2a, "B", "<", "16").
elseedge(case2, case2b).
				\end{program}
				
				
			\subsubsection{Answer-Sets}

				A very simple and obvious translation from \hex programs into answer-sets is to to put all the facts simply as atoms into the answer-set. The above diagram can therefore
				also be implemented as:
			
				\begin{program}[dlvhex]{the above decision diagram as answer-set}{lst:ASDD}
{root(root),
innernode(case1),
innernode(case2),
leafnode(case3, "ClassC"),
leafnode(case1a, "ClassA"),
leafnode(case1b, "ClassB"),
leafnode(case2a, "ClassA"),
leafnode(case2b, "ClassB"),
conditionaledge(root, case1, "A", "<", "10"),
conditionaledge(root, case2, "A", ">", "20"),
elseedge(root, case3),
conditionaledge(case1, case1a, "B", "<", "10"),
elseedge(case1, case1b),
conditionaledge(case2, case2a, "B", "<", "16"),
elseedge(case2, case2b)}
				\end{program}

				
			\subsubsection{RapidMiner XML Format}
			
				Rapidminer\footnote{\url{http://www.rapidminer.com}} is an open-source data mining tool. It uses a priprietary XML file format to store decision trees. This format is also
				supported by the tool introduced in~\ref{sec:Conversion:Converstion}. The details are not relevant for practical work and are skipped therefore. It is only important
				to know that the import and export functionality for this file format is necessary to process RapidMiner classifiers by the \inlinecode{decisiondiagramplugin}.


		\subsection{Conversion}
		\label{sec:Conversion:Converstion}

			For the conversion between the introduced file formats, the plugin installs a tool called \inlinecode{graphconverter}.
			It can be used to translate diagrams in any of the supported file formats into semantically equivalent versions in another format.
			Assume that the diagram is stored in file ``mydiagram.dot". Then the conversion into the according \hex program is done by entering:
			\begin{center}
				\inlinecode{graphconverter dot hex \textless mydiagram.dot \textgreater mydiagram.hex}
			\end{center}

			The result is a set of facts that can be loaded by \dlvhex. After \dlvhex has done it's job, the output will be an answer-set, which is ill-suited for begin read by humans. Thus the plugin
			also supports conversions in the other direction. Assume that \dlvhex' output is stored in file ``answerset.as" (using the \emph{silent mode} such that the output contains
			the \emph{pure} answer-set without any additional information about \dlvhex). Then the conversion is done by:
			\begin{center}
				\inlinecode{graphconverter as dot \textless answerset.dot \textgreater out\_diagram.dot}
			\end{center}

			Between the two converter calls, the diagram is diven as \hex program ``mydiagram.hex" that can be processed by \dlvhex. Even though one can essentially do anything
			with this program that is computable, it is strongly intended to be used as part of the input for a revision task.

			Note that \inlinecode{graphconverter} reads from standard input and writes to standard output.
			It expects either one or two parameters. If one parameter is passed, it can be anything of:
			\begin{itemize}
				\item \verb+--toas+ \\
					Converts a \dotff file into an answer-set.
				\item \verb+--todot+ \\
					Converts an answer-set into a \dotff file.
				\item \verb+--help+ \\
					Displays an online help message.
			\end{itemize}

			Note that \verb+--toas+ and \verb+--todot+ are only abbreviations for commonly used conversions. The more general program call passes two parameters, where the first one
			states the source format and the second one the desired destination format. Both parameters can be anything from the following list.
			\begin{tabularx}{\textwidth}{|X|X|}
				\hline
				Format		& parameter name \\
				\hline
				\dotff graph & \inlinecode{dot} \\
				\hex program & \inlinecode{hexprogram} or \inlinecode{hex} \\
				answer-set & \inlinecode{answerset} or \inlinecode{as} \\
				RapidMiner XML & \inlinecode{rmxml} or \inlinecode{xml} \\
				\hline
			\end{tabularx}
			

	\section{Applying Operators}
	\label{sec:Operators}
	
		This section presupposes familarity with the \textit{mergingplugin}, especially the usage of revision plans and operators. For a detailled description see the according
		documentation.
		
		Listing~\ref{lst:RevisionPlan} shows a typical revision task definition that uses the operators \emph{unfold} and \emph{tobinarydecisiontree}. Note that the decision diagram
		encoded as logic program was directly pasted into the mapping rules. This was only done in order to give the program as one self-contained example. In practice, the mapping
		rules would rather access an external source by the use of external atoms.
		
		\begin{program}[dlvhex]{demonstration of the plugin usage}{lst:RevisionPlan}
[common signature]
predicate: root/1;
predicate: innernode/1;
predicate: leafnode/2;
predicate: conditionaledge/5;
predicate: elseedge/2;

[belief base]
name: kb1;
mapping: "
	root(root).
	innernode(root).
	innernode(v1).
	innernode(v2).
	innernode(v3).
	leafnode(leaf1, class1).
	leafnode(leaf2, class2).
	leafnode(leaf3, class3).
	conditionaledge(root, v1, x, \'<\', y).
	conditionaledge(root, v2, z, \'<\', y).
	elseedge(root, v3).
	conditionaledge(v1, leaf1, a, \'<\', y).
	conditionaledge(v1, leaf2, b, \'<\', y).
	elseedge(v1, leaf3).
	conditionaledge(v2, leaf1, aa, \'<\', y).
	conditionaledge(v2, leaf2, bb, \'<\', y).
	elseedge(v2, leaf3).
	conditionaledge(v3, leaf1, aaa, \'<\', y).
	conditionaledge(v3, leaf2, bbb, \'<\', y).
	elseedge(v3, leaf3).
";

[revision plan]
{
	operator: tobinarydecisiontree;
	{
		operator: unfold;
		{
			kb1
		};
	};
}
		\end{program}
		
		The decisiondiagramplugin ships with several special merging and modification operators for decision diagrams. They expect the belief bases to be sets of facts which were
		generated out of decision diagrams using the \inlinecode{graphconverter} (see~\ref{sec:Conversion:Converstion}). The output will again be a set of facts that encodes a diagram,
		which can be back-converted into a \dotff file. Intermediate results are sets of answer-sets.

		The following subsections describe the operators that are included in the plugin. Note that this is just a very quick and informal description that should enable the user
		to explore the capabilities or adapt operators according to individual needs. For a more detailled and formal description, see the cited thesis.

		\subsection{Modification Operators}
		\label{sec:Operators:Modification}

			\subsubsection{unfold}
			
				The input can be any number of general decision diagrams. The output will contain the same number of diagrams, where each of them has (independently) been converted
				into a tree. This is done by duplication of subtrees if necessary.
			

			\subsubsection{tobinarydecisiondiagram}
			
				The operator is unary, i.e. it works on a single belief base. It expects the input to encode a diagram that is a \emph{tree}, i.e. it contains no sharing of subnodes.
				
				Then the output is again a tree where each node has at most two successors (\emph{binary} tree). This is done by introduction of intermediate nodes.

				
			\subsubsection{orderbinarydecisiontree}
			
				The operator is again unary. It expects it's input to be a \emph{binary} decision \emph{tree}. It's output will a semantically equivalent binary decision tree, where
				on each path from the root to a leaf node, the variables are only queried in lexical ordering.


			\subsubsection{simplify}
			
				The input can be any number of decision diagrams. Each of them is (independently) simplified by some algorithms that will leave the semantics of the diagram unchanged.
				That is, only the structure of the diagram will be modified, which makes them more readable.
			
		
		\subsection{Merging Operators}
		\label{sec:Operators:Merging}
		
			\subsubsection{userpreferences}

				This operator is $n$-ary, i.e. arbitrary many diagrams can be passed. Additionally it expects arbitrary many key-value pairs as parameters, where the keys are ignored
				and the values are of form:
				\begin{center}
					\inlinecode{X \textgreater \textgreater Y} or \inlinecode{X \textgreater n\textgreater Y}
				\end{center}
				where $X$ and $Y$ are the names of class labels (as used in leaf nodes) and $n$ is an integer $\ge 0$. A rule of form \inlinecode{X \textgreater \textgreater Y}
				expresses that ``in doubt, $X$ is preferred to $Y$", whereas \inlinecode{X \textgreater n\textgreater Y} states ``$X$ is preferred to $Y$ if there are at least $n$
				more input diagrams that vote for $X$ than for $Y$".

				The output is a diagram where each domain element is classified according to this rules. Note that the rules are evaluated in top-down manner. That is, the result of
				of a prior rule can be overwritten by a later (applicable!) rule.


			\subsubsection{majorityvoting}
			
				The input can be any number of (general) decision diagrams. The output is a diagram where each domain element is automatically classified by each of the inputs. Then the
				final class label is determined by majority decision. In case that this does not lead to a unique result, the input diagram with the least index forces it's decision.
			

			\subsubsection{avg}

				The operator expects exactly two \emph{ordered binary trees} as input parameter. The result will again be an ordered binary tree of the following form.
				
				For each node from the root to the leafs, if one of the input trees contains condition $X \circ c_1$ and the other one $X \circ c_2$, the resulting tree will contain
				$X \circ \frac{c_1 + c_2}{2}$, i.e. the mean of the comparison value is computed. In case that one of the inputs contains $X \circ c_1$ and the other one $Y \circ c_2$, the
				result will contain $X \circ c_1$ at this position (since $X$ is lexically smaller than $Y$), and the second diagram is incorporated recursivly in both subtrees.
				
				In case of contradicting leaf nodes or incompatible comparison operators (e.g. $X < c_1$ and $X > c_2$), the result is ``unknown".

	\section{Conclusion}
	\label{sec:Conclusion}

		For a more detailed and formal description of the operators see \cite{CR10DD}. For an introduction to the belief merging framework see \cite{CR10BMF}.

	\addcontentsline{toc}{chapter}{References}
	\bibliography{Bibliography}
	\bibliographystyle{plain}

\end{document}
